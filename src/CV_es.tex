\documentclass[11pt,a4paper,sans,spanish]{moderncv}

%% Graphics path
\graphicspath{ {img/} }

%% ModernCV themes
\moderncvstyle{classic}
\moderncvcolor{blue}
\renewcommand{\familydefault}{\sfdefault}
%\nopagenumbers{}

\definecolor{magenta}{RGB}{213, 8, 47}
%\definecolor{color1}{RGB}{243, 25, 83}
%\definecolor{color2}{RGB}{80, 80, 80}

\renewcommand*{\namefont}{\fontsize{33}{38}\mdseries\upshape}

%% Character encoding
\usepackage[utf8]{inputenc}

%% Language
%\usepackage[spanish]{babel}
%\addto\shorthandsspanish{\spanishdeactivate{~<>}}

%% Adjust the page margins
\usepackage[scale=0.81]{geometry}

%% Personal data
\firstname{}
\familyname{}
\address{}{Granada}
\phone{+34 634 511 696}%\\+49 1744 023260\\}
\email{guyik.cgg@gmail.com}
\homepage{https://linkedin.com/in/cglezguerrero}
\photo[100pt][0.0pt]{photo}
\quote{\begin{flushright}La Tierra es la cuna de la humanidad, pero no se puede vivir en la cuna para siempre.
-- Konstantín Tsiolkovski\end{flushright}}

%%-------------------------------------
%% Content
%%-------------------------------------
\begin{document}
%% Custom part of personal data
\begin{minipage}[c]{\textwidth-110pt-0.2em}
    \begin{flushright}
        \textbf{\Huge{Cristian\\\vspace{4pt}González Guerrero}}

        \vspace{4pt}
        \small{53741253-M}
    \end{flushright}
\end{minipage}
\vspace{-5em}

\makecvtitle

%% Working experience
\section{Experiencia Laboral}

\cventry{2021 -- actualidad}
{eesy-innovation Spain}
{Ingeniero de Desarrollo e Investigación}{Granada}{}
{Definición e integración de herramientas colaborativas en el flujo de trabajo de la empresa.\\
Definición de la arquitectura de proyectos firmware.
}


\cventry{2018 -- 2021}
{Real-Time Innovations}
{Ingeniero Software Especialicado en Ciberseguridad}{Granada; Sunnyvale (CA, USA)}{}
{Desarrollo y documentación de características y tests en el núcleo del producto, con énfasis en la experiencia de usuario y robustez.\\
Punto de contacto de seguridad con el equipo de soporte.\\
Responsable del proyecto de documentación del producto Connext DDS Secure.\\
Responsable de la iniciativa de robustez a través de la automatización de detección de warnings y la integración de herramientas de análisis estático.}

\cventry{2018}
{Seven Solutions SL (UGR/OTRI)}
{Ingeniero Responsable de Seguridad}{Granada}{}
{Definición y desarrollo de características de confidencialidad en el protocolo de red.\\
Desarrollo en la interfaz web, con énfasis en su usabilidad.}

\cventry{2016 -- 2017}
{Fundación General UGR-Empresa}
{Desarrollador de Sistemas Empotrados}{Granada}{}
{Integración de biblioteca criptográgrica y protocolo TLS para la seguridad de los datos en dispositivos empotrados para Internet of Things.\\
Integración de protocolos de seguridad para la comunicación con chips de cifrado y firma electrónica.\\
Desarrollo y adaptación de software en sistemas embebidos RTOS.\\
Programación en C de microcontroladores basados en ARM Cortex-M.\\
Redacción y elaboración de documentación.
}

\cventry{2016}
{Infineon Technologies AG}{Ingeniero en prácticas}{Múnich (Alemania)}{}
{Diseño e implementación de sistemas embebidos, desde la concepción y arquitectura del software hasta su implementación y testeo final.\\
Implementación de protocolos de red para su uso por aplicación multitarea.\\
Diseño y auditoría de la seguridad.\\
Diseño gráfico de pósters científico-técnicos y exposiciones.\\
\href{http://secon2016.ieee-secon.org/content/demos-session}{\textbf{Premio a la mejor demo (IEEE SECON 2016)}}, por la exposición del proyecto \textbf{RedFixHop}.
}

\cventry{2016}
{eesy-innovations GmbH}{Ingeniero en prácticas}{Múnich (Alemania)}{}
{Diseño e implementación de sistema de inventariado.\\
Diseño e implementación de sistemas embebidos y de tiempo real.\\
Control de calidad de circuitos impresos.\\
\textbf{Premio especial del departamento Chip Card \& Security (CCS)}, por el proyecto \textbf{Internet of Things \& Security Shields}, en el Application Design Contest 2016, organizado por Infineon.}


%% Languages
\section{Idiomas}

\cvitemwithcomment{Español}{Lengua materna}{~\\~}

\cvitemwithcomment{Inglés}{Nivel avanzado}
{Certificado CAE de Cambridge (equivalente a C1)\\Nivel efectivo: C1}

\cvitemwithcomment{Francés}{Nivel avanzado}
{Idioma hablado durante el programa Erasmus\\Nivel efectivo: B2}

\cvitemwithcomment{Alemán}{Nivel básico}
{Curso intensivo durante la estancia en Múnich\\Nivel efectivo: A1}


%% Academic training
\section{Formación Académica}

\cventry{2021}
{Programa de Estudios Espaciales del Hemisferio Sur}
{International Space University}{}{}
{Trabajo Final:
Space Assets and Technology for Bushfire Management}

\cventry{2016 -- 2018}
{Máster en Ciencia de Datos e Ingeniería de Computadores}
{Universidad de Granada}{}{}
{Trabajo Fin de Máster:
Seguridad de los Datos en Sistemas Empotrados (10 sobre 10)}

\cventry{2010 -- 2015}
{Grado en Ingeniería de Tecnologías de Telecomunicación, especialidad en Sistemas Electrónicos}
{Universidad de Granada}{}{}
{Nota media: 7,479 (sobre 10).
Trabajo Fin de Grado:
Sistema Digital de Extracción del Ritmo Cardíaco Fetal mediante Filtrado Adaptativo (9,8 sobre 10).}

\cventry{2013 -- 2014}
{Movilidad con programa Erasmus}
{Université Catholique de Louvain}{Louvain-la-Neuve}{Bélgica}
{Colaboración con la asociación por la divulgación de la lengua de signos \emph{KAP Signes.}}


%% Complementary training
\section{Formación Complementaria}

\cventry{2019 -- actualidad}
{Team Leader en \href{https://orbitando.space}{Orbitando}}
{}{Granada}{}
{Co-Fundador de la asociación. Definición de objetivos y planificación. Coordinación con otras entidades. Definición y realización de actividades de comunicación científica. Implementación de la web.}

\cventry{2018}
{Voluntario en el Space Studies Program}
{International Space University}{TU Delft}{Países Bajos}
{}

\cventry{2016}
{Machine learning}
{Universidad de Stanford}{Coursera}{}
{}

\cventry{2014 -- 2015}
{III Curso de Emprendedores/as Universitarios/as}
{Aula Andalucía Emprende}{}{}
{Obtenida la \textbf{mención a la 3ª mejor idea empresarial} por el proyecto \emph{Odisey}. }

\cventry{2014 -- 2016}
{Presidente de SCOPE Junior Empresa}
{}{}{}
{Representación legal de la asociación. Instructor de diversos cursos y talleres.
}

\cventry{2013}
{Scratch: Iniciación a la Programación para Enseñanza Primaria y Secundaria. II~Edición}
{Fundación General UGR-Empresa, CEVUG}{}{}
{Obtenida la calificación de sobresaliente.}

\cventry{2012}
{IV Curso de Oratoria y Retórica}
{Agrupación Centro de Cultura}{}{}
{Obtenida la calificación de sobresaliente.}

\section{Datos de Interés}
\href{http://ieeexplore.ieee.org/stamp/stamp.jsp?tp=&arnumber=7733012&isnumber=7732953}{Participación en congresos internacionales en el campo de redes inalámbricas.}\quad Conocimientos en sistemas de captación, tratamiento y extracción de información de señales de origen biológico.\quad
%\protect\\[0.3em]
Conocimientos avanzados en modelado y simulación de sistemas.\quad
Conocimientos avanzados en tratamiento digital de señales.\quad
%\protect\\[0.3em]
Conocimientos en diseño gráfico.\quad
Permiso de conducir B. \quad{} Trabajo en equipo. \quad{} Facilidad para hablar en público. \quad{} Gestión de proyectos. \quad{} Interés en el arte y la cultura. \quad{} Interés en la enseñanza.
%\protect\\[0.3em]


\end{document}
