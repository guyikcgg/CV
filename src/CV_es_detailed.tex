\documentclass[11pt,a4paper,sans,spanish]{moderncv}

%% Graphics path
\graphicspath{ {img/} }

%% ModernCV themes
\moderncvstyle{classic}
\moderncvcolor{blue}
\renewcommand{\familydefault}{\sfdefault}
%\nopagenumbers{}

\definecolor{magenta}{RGB}{213, 8, 47}
%\definecolor{color1}{RGB}{243, 25, 83}
%\definecolor{color2}{RGB}{80, 80, 80}

\renewcommand*{\namefont}{\fontsize{33}{38}\mdseries\upshape}

%% Character encoding
\usepackage[utf8]{inputenc}

%% Language
%\usepackage[spanish]{babel}
%\addto\shorthandsspanish{\spanishdeactivate{~<>}}

%% Adjust the page margins
\usepackage[scale=0.8]{geometry}

%% Color links
\newcommand{\colorhref}[3][color1]{\href{#2}{\color{#1}{\underline{#3}}}}

%% Personal data
\firstname{}
\familyname{}
\address{}{Granada}
\phone{+34 634 511 696}
\email{guyik.cgg@gmail.com}
\homepage{linkedin.com/in/cglezguerrero}
\photo[100pt][0.0pt]{foto_rec}
\quote{Graduado en Ingeniería de Tecnologías de la Telecomunicación con experiencia en programación de sistemas embebidos.}

%%-------------------------------------
%% Content
%%-------------------------------------
\begin{document}
%% Custom part of personal data
\begin{minipage}[c]{\textwidth-110pt-0.2em}
    \begin{flushright}
        \textbf{\Huge{Cristian\\\vspace{4pt}González Guerrero}}

        \vspace{4pt}
        \small{53741253-M}
    \end{flushright}
\end{minipage}
\vspace{-5em}

\makecvtitle

%% Working experience
\section{Experiencia Laboral}

\cventry{jun 2017 -- sep 2017}
{Infineon Technologies AG (a través de un contrato con la Universidad de Granada)}
{Desarrollador de Sistemas Empotrados}{Granada}{}
{\emph{Llevando a  cabo las tareas de:} Implementación de protocolos seguros para la comunicación con chips de cifrado y firma electrónica.
Desarrollo y adaptación de software en sistemas embebidos RTOS.
Programación en C de microcontroladores basados en ARM Cortex-M.
}

\cventry{dic 2016 -- feb 2017}
{eesy-innovations GmbH (a través de la Fundación General UGR-Empresa)}
{Desarrollador de Sistemas Empotrados}{Granada}{}
{\emph{Llevando a  cabo las tareas de:} Implementación de protocolo TLS para la seguridad de los datos en dispositivos empotrados para Internet of Things.
Redacción y elaboración de documentación.
}

\cventry{jun 2016 -- sep 2016}
{Infineon Technologies AG}{Ingeniero en prácticas (estancia de 4 meses)}{Múnich}{Alemania}
{\emph{Llevando a  cabo las tareas de:} Diseño e implementación de sistemas embebidos, desde la concepción y arquitectura del software hasta su implementación y testeo final.\\
Implementación de protocolos de red para su uso por aplicación multitarea.
Diseño y auditoría de la seguridad.
Diseño gráfico de pósters científico-técnicos y exposiciones.
\colorhref{http://secon2016.ieee-secon.org/content/demos-session}{Premio a la mejor demo (IEEE SECON 2016)}, por la exposición del proyecto \textbf{RedFixHop}.
}

\cventry{mar 2016 -- may 2016}
{eesy-innovations GmbH}{Ingeniero en prácticas (estancia de 3 meses)}{Múnich}{Alemania}
{\emph{Llevando a  cabo las tareas de:} Diseño e implementación de sistema de inventariado.
Diseño e implementación de sistemas embebidos y de tiempo real.
Control de calidad de circuitos impresos.
\textbf{Premio especial del departamento Chip Card \& Security (CCS)}, por el proyecto \textbf{Internet of Things \& Security Shields}, en el Application Design Contest 2016, organizado por Infineon.}


%% Languages
\section{Idiomas}

\cvitemwithcomment{Español}{Lengua materna}{~\\~}

\cvitemwithcomment{Inglés}{Nivel avanzado}
{Certificado CAE de Cambridge (equivalente a C1)\\Nivel efectivo: C1}

\cvitemwithcomment{Francés}{Nivel avanzado}
{Idioma hablado durante el programa Erasmus\\Nivel efectivo: B2}

\cvitemwithcomment{Alemán}{Nivel básico}
{Curso intensivo durante la estancia en Múnich\\Nivel efectivo: A1}


%% Academic training
\section{Formación Académica}

\cventry{oct 2016 -- \\jul 2018'}
{Máster en Ciencia de Datos e Ingeniería de Computadores}
{Universidad de Granada}{}{}
{'Formación actualente en curso}

\cventry{oct 2010 -- sep 2015}
{Grado en Ingeniería de Tecnologías de Telecomunicación, especialidad en Sistemas Electrónicos}
{Universidad de Granada}{}{}
{Nota media: 7,479 (sobre 10).
Trabajo Fin de Grado:
Sistema Digital de Extracción del Ritmo Cardíaco Fetal mediante Filtrado Adaptativo (9,8 sobre 10).}

\cventry{sep 2013 -- ago 2014}
{Movilidad con programa Erasmus}
{Université Catholique de Louvain}{Louvain-la-Neuve}{Bélgica}
{Colaboración con la asociación por la divulgación de la lengua de signos \emph{KAP Signes.}}


%% Complementary training
\section{Formación Complementaria}

\cventry{jul 2016 -- ago 2016}
{Machine learning}
{Universidad de Stanford}{Coursera}{}
{}

\cventry{nov 2014 -- feb 2015}
{III Curso de Emprendedores/as Universitarios/as}
{Aula Andalucía Emprende}{}{}
{Obtenida la \colorhref{http://empleo2.ugr.es/noticia/elevator-pitch-presenta-tu-idea-empresarial}{mención a la 3ª mejor idea empresarial} por el proyecto \emph{Odisey}. }


%% Technical skills
\section{Habilidades Técnicas}

\subsection{Informática y Telemática}
{Conocimientos en fundamentos de sistemas operativos y gestión de recursos.}\quad
{Administración avanzada de sistemas informáticos.}\quad
%\protect\\[0.3em]
{Conocimientos en bases de datos y tecnologías web.}\quad
%\protect\\[0.3em]
{Programación avanzada de soluciones en niveles cercanos a máquina.}\quad
%\protect\\[0.3em]
{Diseño, análisis, gestión y mantenimiento de redes y protocolos.}\quad
%\protect\\[0.3em]
{Conocimientos en virtualización y despliegue de servidores.}\quad
{Experiencia con metodologías ágiles, control de versiones y sistemas de ticketing.}\quad
{Experiencia con protocolos de comunicación IoT.}

\begin{center}
\textcolor{cyan}{
Linux \quad{} Minix \quad{} MS~Windows \quad{} C/C++ \quad{} SVN \quad{} Git \quad{}
Github \quad{} Bitbucket \quad{} Jira \quad{} TCP/IP \quad{} HTTP \quad{}
MQTT \quad{} SSL/TLS \quad{} Cisco
}
\end{center}

\subsection{Desarrollo web}
{Desarrollo de aplicaciones web, desde la concepción y diseño de interfaces hasta el testeo final.}\quad
{Desarrollo frontend y backend con múltiples tecnologías.}\quad
%\protect\\[0.3em]
{Diseño de bases de datos relacionales y no relacionales.}\quad
%\protect\\[0.3em]
{Implementación y personalización de CMS.}\quad
%\protect\\[0.3em]
{Despliegue de servidores en redes de telecomunicaciones.}

\begin{center}
\textcolor{cyan}{
HTML5 \quad{} CSS3 \quad{} Bootstrap \quad{} JavaScript \quad{} jQuery \quad{}
Ajax \quad{} SQL \quad{} MongoDB \quad{} JSON \quad{} XML \quad{} PHP \quad{}
NodeJS \quad{} Apache2 \quad{} Django
}
\end{center}

\subsection{Otras habilidades y logros}
\colorhref{http://ieeexplore.ieee.org/stamp/stamp.jsp?tp=&arnumber=7733012&isnumber=7732953}{Participación en congresos internacionales.}\quad
Conocimientos en ciencia de datos.
Conocimientos avanzados en modelado y simulación de sistemas.\quad
Conocimientos avanzados en tratamiento digital de señales.\quad
\colorhref{https://valeoinnovationchallenge.valeo.com/news/interviews/introducing-team-ecopilot}{Semifinalista en Valeo Innovation Challenge 2017},
con el proyecto \colorhref{http://myecopilot.com}{ECOpilot}.\quad
Conocimientos en sistemas de captación, tratamiento y extracción de información de señales de origen biológico y biosanitario.\quad
%\protect\\[0.3em]
Conocimientos avanzados en electrónica y robótica.\quad
%\protect\\[0.3em]
Conocimientos avanzados en ofimática.\quad
%\protect\\[0.3em]
Conocimientos en diseño gráfico.\quad
Permiso de conducir B. \quad{} Carácter fuertemente emprendedor. \quad{} Trabajo en equipo. \quad{} Facilidad para hablar en público. \quad{} Gestión de proyectos. \quad{} Interés en el arte y la cultura. \quad{} Interés en la enseñanza.
%\protect\\[0.3em]

\begin{center}
\textcolor{cyan}{
Matlab \quad{} GNU~Octave \quad{} R \quad{} Mathematica \quad{} LabView \quad{} wxMaxima \quad{} LaTeX \quad{} LibreOffice \quad{} Gimp \quad{} Photoshop \quad{} Inkscape \quad{} AutoCAD
}
\end{center}


\end{document}
